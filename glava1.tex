\section{Организационно-управленческая характеристика}

ООО Сергинское основано в 2010 году путем реорганизации в форме преобразования. 

Адрес организации: 617258б Пермский кра, Сивинский район, село Серьгино, ул. Центральная, 2.

Директор: Ужегов Борис Анатольевич.

Уставный капитал компании 821 тыс. руб. Количество учредителей --- 7 человек.

Основным видом деятельности является выращивание однолетних культур. Среди дополнительных видов деятельности:  выращивание многолетних культур, животноводство, производство молочной продукции и др.

Компания является сельхозтоваропроизводителем, племенным репродуктором по разведению герефордов. На ферме содержится более 700 голов КРС молочного и мясного направления. 

По итогам 2016 года ООО Сергинское заняло четвертое место в Сивинском районе по объему производства молока --- 1741 тонн. Продуктивность коров составила в год 5270 кг/гол., поголовье коров молочного направления на начало 2018 --- 467 голов. Численность работников в 2017 г. --- 76 человек.

Объем субсидий, полученных предприятием в 2016 г., составил 13,33 млн. рублей (8,52 млн. руб. предоставлено за счет средств краевого бюджета, 4,81 млн. руб. --- за счет средств федерального бюджета).

В 2017 году компания завершила реализацию инвестиционного проекта по введению в эксплуатацию модернизированной молочно-товарной фермы на 200 голов. Ферма получила роботизированный коровник на два робота голландской фирмы Lely Astronaut A4  с молочным танком Muller 4000.

Общая стоимость проекта по сводному сметному расчету составила 28,5 млн. рублей. Сельхозтоваропроизводитель реализовал инвестпропроект за счет собственных и заемных средств.

Проект предусматривает увеличение продуктивности коров до 6000 кг на голову. Ожидается, что в 2018 году сельхозпредприятие выйдет на проектную мощность.

Основные показатели деятельности приведены в таблице \ref{osn-pokazat}.

Исходя из данных таблицы \ref{osn-pokazat} 

Проведем анализ состава и структуры товарной продукции компании \ref{struktura-prod}.















































